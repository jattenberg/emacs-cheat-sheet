%%%%%%%%%%%%%%%%%%%%%%%%%%%%%%%%%%%%%%%%%
% Cheatsheet
% LaTeX Template
% Version 1.0 (12/12/15)
%
% This template has been downloaded from:
% http://www.LaTeXTemplates.com
%
% Original author:
% Michael Müller (https://github.com/cmichi/latex-template-collection) with
% extensive modifications by Vel (vel@LaTeXTemplates.com)
%
% License:
% The MIT License (see included LICENSE file)
%
%%%%%%%%%%%%%%%%%%%%%%%%%%%%%%%%%%%%%%%%%

%----------------------------------------------------------------------------------------
%	PACKAGES AND OTHER DOCUMENT CONFIGURATIONS
%----------------------------------------------------------------------------------------

\documentclass[11pt]{scrartcl} % 11pt font size

\usepackage[utf8]{inputenc} % Required for inputting international characters
\usepackage[T1]{fontenc} % Output font encoding for international characters

\usepackage[margin=0pt, landscape]{geometry} % Page margins and orientation

\usepackage{graphicx} % Required for including images

\usepackage{color} % Required for color customization
\definecolor{mygray}{gray}{.75} % Custom color

\usepackage{url} % Required for the \url command to easily display URLs

\usepackage{amsmath,amssymb,stmaryrd}

\usepackage[ % This block contains information used to annotate the PDF
colorlinks=false, 
pdftitle={Emacs Tips}, 
pdfauthor={Josh Attenberg}, 
pdfsubject={A selection of useful emacs tips}, 
pdfkeywords={Emacs, Cheatsheet, Clojure}
]{hyperref}

\setlength{\unitlength}{1mm} % Set the length that numerical units are measured in
\setlength{\parindent}{0pt} % Stop paragraph indentation

\renewcommand{\dots}{\ \dotfill{}\ } % Fills in the right amount of dots

\newcommand{\command}[2]{#1~\dotfill{}~#2\\} % Custom command for adding a shorcut

\newcommand{\sectiontitle}[1]{\paragraph{#1} \ \\} % Custom command for subsection titles

%----------------------------------------------------------------------------------------

\begin{document}

\begin{picture}(297,210) % Create a container for the page content

%----------------------------------------------------------------------------------------
%	TITLE SECTION 
%----------------------------------------------------------------------------------------

\put(10,200){ % Position on the page to put the title
\begin{minipage}[t]{210mm} % The size and alignment of the title
\section*{Emacs Shortcuts and Useful Commands} % Title
\end{minipage}
}

%----------------------------------------------------------------------------------------
%	FIRST COLUMN SPECIFICATION
%----------------------------------------------------------------------------------------

\put(10,180){ % Divide the page
\begin{minipage}[t]{85mm} % Create a box to house text

%----------------------------------------------------------------------------------------
%	HEADING ONE
%----------------------------------------------------------------------------------------

\sectiontitle{Window Management}
			
\texttt{C} = Ctrl key\
\texttt{M} = meta key (usually escape)\

\command{\texttt{C-x + o}}{Switch active window, move cursor to another window}
\command{\texttt{C-x + 1}}{Close all other windows, leaving only the active window open. (doesn't close buffers or lose work)}
\command{\texttt{C-x + 2}}{Split window above \& below}
\command{\texttt{C-x + 3}}{Split window left and right}
\command{\texttt{C-x + 0}}{Close current window}

%----------------------------------------------------------------------------------------
%	HEADING TWO
%----------------------------------------------------------------------------------------				
			
\sectiontitle{Files \& Buffers}
			
\command{\texttt{C-x C-f}}{Open a file}
\command{\texttt{C-x C-s}}{Safe a buffer to a file}
\command{\texttt{C-x b}}{Switch or open buffer}

%----------------------------------------------------------------------------------------
%	HEADING THREE
%----------------------------------------------------------------------------------------	

\sectiontitle{Movement}

\command{\texttt{C-a}}{Move to the beginning of the line}
\command{\texttt{M-m}}{Move to the first non-whitespace character in the line}
\command{\texttt{C-e}}{Move to the end of the line}
\command{\texttt{C-f}}{Move forward one character}
\command{\texttt{C-b}}{Move backward one character}
\command{\texttt{C-a}}{Move to the beginning of the line}
\command{\texttt{M-f}}{Move forward one word}
\command{\texttt{M-b}}{Move backward one word}
\command{\texttt{C-s}}{Regex search within the current buffer, move to matches (\texttt{C-s} again to move to subsequent matches}
\command{\texttt{C-r}}{Same as \texttt{C-s} but searches in reverse}
\command{\texttt{M-<}}{Move to the beginning of the buffer}
\command{\texttt{M->}}{Move to the end of the buffer}
\command{\texttt{M-g g}}{Go to the specified line number}

%----------------------------------------------------------------------------------------

\end{minipage} % End the first column of text
} % End the first division of the page

%----------------------------------------------------------------------------------------
%	SECOND COLUMN SPECIFICATION 
%----------------------------------------------------------------------------------------

\put(105,180){ % Divide the page
\begin{minipage}[t]{85mm} % Create a box to house text

%----------------------------------------------------------------------------------------
%	HEADING FOUR
%----------------------------------------------------------------------------------------

\sectiontitle{Kill Ring}

\command{\texttt{C-space}}{Set a mark denoting the beginning of a region}
\command{\texttt{M-w}}{Copy highlighted region to kill ring}
\command{\texttt{C-w}}{Cut (kill) highlighted region to kill ring (removing from buffer}
\command{\texttt{C-k}}{Cut the entire line after the cursor to the kill ring}
\command{\texttt{M-b}}{Cut word after cursor to the kill ring}
\command{\texttt{C-y}}{Paste (yank) contents of the kill ring to the buffer}
\command{\texttt{M-y}}{Cycle through the contents of the kill ring}

%----------------------------------------------------------------------------------------
%	HEADING FIVE
%----------------------------------------------------------------------------------------				
					
\sectiontitle{Editing} % Heading five

\command{\texttt{Tab}}{Indent line}
\command{\texttt{C-j}}{Create a new line and then indent}
\command{\texttt{M-/}}{Cycles through all expansions of the text before this point}
\command{\texttt{M-\textbackslash}}{Delete all spaces and tabs around this point}

%----------------------------------------------------------------------------------------

\sectiontitle{Help} % Heading six

\command{\texttt{C-h k (key binding)}}{describe the function bound to the specified key binding}
\command{\texttt{C-h f}}{Describe function}

%----------------------------------------------------------------------------------------


\end{minipage} % End the second column of text
} % End the second division of the page

%----------------------------------------------------------------------------------------
%	THIRD COLUMN SPECIFICATION 
%----------------------------------------------------------------------------------------

\put(200,180){ % Divide the page
\begin{minipage}[t]{85mm} % Create a box to house tex

%----------------------------------------------------------------------------------------
%	Clojure
%----------------------------------------------------------------------------------------

\sectiontitle{\texttt{cider}}
Open with: \texttt{M-x cider-jack-in}\\
see: \url{https://github.com/clojure-emacs/cider}

\command{\texttt{C-x C-e}}{Evaluate the preceding command}
\command{\texttt{C-u C-x C-e}}{Evaluate the preceding command, print results after the cursor in the current buffer}
\command{\texttt{C-c M-n}}{Switch to the namespace of the current buffer}
\command{\texttt{C-c C-k}}{Compile the current buffer}
\command{\texttt{M-.} and \texttt{M-,}}{Navigate to source code for the symbol under point and return to the original buffer}
\command{\texttt{C-c C-d C-a}}{Find arbitrary text across function names and documentation}
\command{\texttt{C-up} and \texttt{C-down}}{Navigate through repl history}
\command{\texttt{C-enter}}{Close parens and evaluate}



\vspace{\baselineskip} % Whitespace before the next section

%----------------------------------------------------------------------------------------
%	LINKS AND INFORMATION
%----------------------------------------------------------------------------------------

\sectiontitle{Links and information}

\url{https://github.com/clojure-emacs/cider}

\url{https://sachachua.com/blog/wp-content/uploads/2013/05/How-to-Learn-Emacs8.png}

\url{https://www.gnu.org/software/emacs/manual/html_node/emacs/}

\url{https://www.masteringemacs.org/reading-guide}

\url{https://www.gnu.org/software/emacs/refcards/pdf/refcard.pdf}

%----------------------------------------------------------------------------------------
%	FOOTNOTE
%----------------------------------------------------------------------------------------

\vspace{\baselineskip}
\linethickness{0.5mm} % Thickness of the footer line
{\color{mygray}\line(1,0){30}} % Print the line with a custom color

\footnotesize{
Created by Josh Attenberg, 2019\\ 
\url{https://github.com/jattenberg/emacs-cheat-sheet}\\
				
Released under the MIT license.
}

%----------------------------------------------------------------------------------------

\end{minipage} % End the third column of text
} % End the third division of the page
\end{picture} % End the container for the entire page

%----------------------------------------------------------------------------------------

\end{document}